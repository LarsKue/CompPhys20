\documentclass[11pt, a4paper]{article}
\usepackage{listings}
%\usepackage{bera}
\lstset{basicstyle=\ttfamily, mathescape=true, breaklines=true}

\title{Homework 5}
\date{Due May 29, 2020}
\author{Daniel Kreuzberger, Lars Kuehmichel, David Weinand}

\begin{document}
\maketitle
\section{Expression for Gaussian Elimination}
We have $Ax = b$ with a tridiagonal matrix $A$ of dimension $N \times N$, so we can start directly without having to reorder the rows. The first row of $A$ will stay the same, so we do an iteration over $N-1$ other rows:

\begin{lstlisting}
for i := 1 to N-1:
    factor = A$_{i+1, i}$ / A$_{i, i}$
    # calculate the factor with which we have to multiply the previous row and add it to the next row to eleminate the left off-diagonal element
    b$_{i+1}$ = b$_{i+1}$ - factor * b$_{i}$
    # update the vector b so the solution stays the same
    for k := 1 to N:
        A$_{i+1, k}$ = A$_{i+1, k}$ - factor * A$_{i, k}$
        # update the row i+1
\end{lstlisting}

With this algorithm, we get an upper right triangular matrix $A$ and an updated vector $b$. By doing backward substitution, we can get the solution of the linear equation system.
\newpage
\section{Backward Substitution}
We have $Ax = b$ with an upper right triangular matrix $A$ of dimension $N \times N$. We have to start with:
\begin{lstlisting}
x$_{N}$ = b$_{N}$ / A$_{N, N}$

for i := 1 to N-1:
    for k := N-(i-1) to N:
        n$_{N-i}$ = n$_{N-i}$ - A$_{N-i, k}$ * x$_{k}$
        # since the off-diagonal elements are not zero, we have to subtract them times the solution x
        x$_{N-i}$ = b$_{N-1}$ / A$_{N-i, N-i}$
\end{lstlisting}

with this algorithm, we get the solution vector $x$.


\end{document}